
Il mio compito nello sviluppo dell'applicazione è stato quello 
di creare un prototipo iniziale avendo a disposizione un mock up creato con
proto.io\cite{protoio} e una serie di requisiti essenziali.

\section{Requisiti essenziali}

La base di partenza di QIX sono state delle funzionalità essenziali e 
sonstanzialmente molto difficili da inserire in una versione dell'app già avanzata.
È stati quindi deciso di creare un prototipo di partenza avente i seguenti requisiti:

\begin{itemize}
    \item {
        L'applicazione deve avere la capacità di reindirizzare 
        dinamicamente l'utente in sezioni programmate differenti in base a un constesto.
        Analizzando il requisito mi sono posto diverse domande:
        \textit{
            Cosa significa dinamicamente?
            Cosa sono delle sezioni programmate?
            Quale contesto?
        }
    }
    \item {
        La funzionalità \textbf{QIX Shake} dell'applicazione
        deve essere disponibile in una qualunque sezione o vista si trovi l'utente
        e la sua funzione deve essere determinata dal contesto attuale;
    } 
    \item {
        L'intera applicazione dev'essere progettata in modo tale da presentare all'utente
        delle \textbf{animazioni interattive} in stile CardView\cite{cardview} disponibili in 
        qualunque sezione o vista in cui si trovi l'utente e definite dal contesto attuale;

        Le animazioni in questione devono essere progettate in pagine, in cui ogni pagina può contenere 
        più CardView. L'utente vedrà in un determinato momento una e soltanto una pagine.

        Queste viste devono essere trascinabili dall'utente e quando egli usa una forza
        di trascinamento superiore a un valore di soglia devono cadere attraverso una gravità;
        
        Tale gravità finirà con la fine dell'animazione o l'apparizione della nuova pagina;
    }
    \item {
        L'applicazione deve supportare tre diversi stati o modalità di autenticazione:
        \begin{enumerate}
            \item\textbf{Trial Mode}: l'utente è anonimo, esiste solo un id per tenere traccia dei suoi QIX coins.
            \item\textbf{Signed Mode}: l'utente ha inserito il numero di telefono e il suo genere;
            \item \textbf{Pro Mode}: l'utente aggiunge dei dati su se stesso o collega il suo account a dei social media come Facebook, Google o Instagram;
        \end{enumerate}
        Si nota facilmente che non esiste una stato in cui l'utente non è registrato: questo perchè
        per tenere traccia dei suoi QIX coins e di altri dati utili è necessario avere una riferimento all'utente;
    }
    \item {
        L'applicazione deve poter essere avviata dinamicamente
        attraverso dei \textbf{Deep Links}\cite{deeplinks};
        E deve essere in grado di gestirli in base al contesto dell'utente;
    }
\end{itemize}