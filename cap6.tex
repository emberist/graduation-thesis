
Per la progettazionde dell'autenticazione si è voluto ricorrere a qualcosa di 
già pronto per velocizzare i tempi di sviluppo e quindi evitare di progettare e implementare un sistema 
complesso come la gestione di utenti non autenticati (Guest).
Cercando in rete abbiamo optato per l'utilizzo di \textbf{Firebase}\cite{firebase}

\section{Firebase Authentication}

Firebase è una piattaforma
di sviluppo per applicazione mobili e web acquisita di Google nel 2014. 
Uno dei motivi fondamentali per cui è stata scelta è la grande quantità di servizi utili come
analisi dei Crash, tracking della naviazione degli utenti, DynamicLinks (si veda sezione~\ref{section:dynamiclinks})
e ovviamente l'autenticazione.

\section{Tipologie di autenticazione}

Esistono 3 tipologie di autenticazione come definito nel requisito, ognuna della quali è un'estensione
di quella precedente: un utente che apre per la prima volta l'applicazione verrà subito registrato 
come utente anonimo e quindi in \textbf{Trial Mode}, se poi decide di effettuare la registrazione può inserire 
il suo numero di telefono o la sua email e attraverso Firebase l'utente anonimo evolverà in un utente con più
informazioni, in \textbf{Signed Mode}.

Successivamente ci sarà nell'applicazione una sezione specifica dove l'utente potrà inserire nuovi dati come nome, cognome e data di nascita
o semplicemente potrà linkare un social media come Facebook, Google o Instagram. In questo caso l'utente evolverà nuovamente 
e arriverà nella \textbf{Pro Mode}.

Il pacchetto Firebase Auth infatti include tutte queste funzionalità e attraverso delle semplici API
è possibile utilizzarlo come segue:

Inizializzazione di Firebase nel file di inizializzazione dell'applicazione ossia l'AppDelegate.swift

\begin{minted}{swift}
    FirebaseApp.configure()
\end{minted}

Creazione di un utente anonimo

\begin{minted}{swift}
    Auth.auth().signInAnonymously() { (authResult, error) in
        // ...
    }
\end{minted}

Collegamento di un account anonimo con numero di telefono o email

\begin{minted}{swift}
    // L'oggetto user è sempre disponibile in firebase e
    // salvato nel keychain di iOS
    let user = Auth.auth().currentUser

    // Creazione dell'oggetto credenzial da collegare
    // all'utente attuale (anonimo o non)
    let credential = EmailAuthProvider.credential(withEmail: email,
        password: password)

    // Linking dei due account
    user?.link(with: credential) { (authResult, error) in
        // ...
    }
\end{minted}

Il linking con un qualunque social media è simile all'esempio riportato
qui sopra, cambia soltanto il Provider delle credenziali
