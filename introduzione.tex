\normalsize
Durante tirocinio presso \textbf{Urbana Smart Solutions srl}\cite{urbanasmartsolutions}
ho iniziato a implementare un'applicazione per la fidelizzazione e rewarding di un target di utenti specifico. 
Il principali target di clienti a cui mira l'applicazione sono FMCG\footnote{Fast Moving Consumer
Goods}, ossia società che vendono beni di consumo a basso
costo e molto velocemente.

Tali compagnie attraverso i loro prodotti possono creare diverse
tipologie di eventi e gli utenti dell’applicazione possono accedervi
e vincere dei punti denominati \textbf{QIX coins} con cui comprare o avere degli
sconti sui beni venduti.

% Esistono diverse modalità in cui un utente può accadere a tali eventi:
% \begin{itemize}
%     \item Usando la funzione “shake” dello smartphone in determinati contesti;
%     \item Usando specifiche funzioni come la G'morning Challenge o la ruota della fortuna;
% \end{itemize}

L’elemento cardine dell’app è il \textbf{QIX Shake} ossia l’attivazione di un particolare servizio
che dipende dal contesto attuale dell'utente e il tipo di offerta che viene selezionata.
Tale servizio dà agli utenti la possibilità di vincere dei QIX coins agitando lo smartphone.
I principali servizi di shake dell'applicazione sono:

\begin{enumerate}
    \item\textbf{TV Shake}: agitando lo smartphone inizierà un'analisi dei dati del microfono
    allo scopo di trovare un particale \textbf{watermark} inserito in campagnie publicitarie televisive;
    \item\textbf{Read Shake}: all'utente viene proposta la lettura di contenuti o questionari in cambio di QIX coins;
    \item\textbf{Video Shake}: a seguito della visualizzazione di uno video l'utente viene premiato con dei punti;
    \item\textbf{Scan shake}: dopo aver scannerizzato il barcode di un prodotto (Es. al supermercato) all'utente verrano assegnati dei punti;
    \item\textbf{Receipt Shake}: dopo aver scannerizzato uno scontrino di acquisto di prodotti partner delle FCMG;
    \item\textbf{Stadium Shake}: inierà anche in questo caso l'analisi del suono esterno per la ricerca di eventuali watermark
    generati in un'audio durante delle partite allo stadio;
\end{enumerate}


Nel capitolo 1 ho enunciato l'obiettivo principale del mio lavoro persso l'azienda e
i requisiti principali e obbligatori che l'applicazione finale dovrà possedere.
Nel capitolo 2 ho introdotto gli elementi nativi di iOS che sono andato ad utilizzare in larga scala
durante l'implementazione. In tutti i capitoli successivi descrivo capitolo per capitolo tutti i requisti
base descritti nel capitolo 1.