\normalsize
L’obiettivo principale del tirocinio presso \textbf{Urbana Smart Solutions srl}\cite{urbanasmartsolutions} è stata la creazione di un'applicazione iOS atta alla fidelizzazione e al rewarding di un target di utenti specifico. I principali clienti di QIX sono delle FMCG\footnote{Fast Moving Consumer Goods}, ossia delle compagnie che vendono beni di consumo a basso costo e molto velocemente.

Tali compagnie attraverso i loro prodotti possono creare diverse tipologie di eventi e gli utenti dell’applicazione possono accedervi e vincere dei QIX coins, ossia dei punti con cui comprare o avere degli sconti sui beni venduti.

Esistono diverse modalità in cui un utente può accadere a tali eventi:
\begin{itemize}
    \item Usando la funzione “shake” dello smartphone in determinati contesti;
    \item Usando specifiche funzioni come la G'morning Challenge o la ruota della fortuna;
\end{itemize}

L’elemento cardine dell’app è il “QIX Shake” ossia l’attivazione di un particolare servizio e la possibilità di vincere dei QIX coins agitando lo smartphone. Tale funzionalità si divide in diverse tipologie:

\begin{enumerate}
    \item\textbf{TV Shake}: un qualsiasi utente di QIX potrà tentare di vincere dei punti
    \item\textbf{Read Shake}: i QIX coins vengono consegnati una volta letto una sorta di questionario
    \item\textbf{Video Shake}: Dopo aver guardato un video
    \item\textbf{Scan shake}: dopo aver scannerizzato il barcode di un prodotto delle FMCG
    \item\textbf{Receipt Shake}: dopo aver scannerizzato uno scontrino
    \item\textbf{Stadium Shake}: Ascoltando della musica negli stadi con una watermark non udibile dall'uomo
\end{enumerate}

L'obiettivo principale è stato quindi quello di progettare
un prototipo iniziale che rispettasse determinati requisiti

