
Un'altra interessante funzione di Firebase sono i \textbf{Dynamic Links}
ossia dei semplici URL generati dalla console o direttamente dall'applicazione
che consentono la creazione semplicata di particolari \textbf{DeepLink}.

In particolare i DeepLink in Android e iOS sono degli URl che vengono registrati nell'applicazione
e non fanno altro che evitare l'apertura del browser rispetto a quella dell'applicazione.

Questo permette agli utenti di aprire direttamente l'applicazione in questione
attraverso lo smartphone o il Computer attraverso il browser, ma trasportando dei dati utili all'azienda.



In particolare nel mio caso mi è stato chiesto di implementare un sistema di inviti, per cui 
un utente può invitarene un altro attraverso questi links ed è necessario tracciare
l'utente che ha inviato l'invito per iviargli un premio.

Per risolvere il problema abbiamo prima implementato i Dynamic Links attraverso le API de
di Firebase e successivamente nella creazione del Link iniettiamo dell'URL un id univoco che identifica l'utente
che crea il link di invito.

\subsection{Generazione di un DynamicLink}

Attraverso Firebase risulta molto semplice generare il DynamicLink,
l'unica cosa che ci serve è l'id dell'utente

\begin{minted}{swift}
let userId = Configuration.shared.fetchUserId() 

// DeepLink registrato nell'app
let targetLink = URL(string: "https://myqix.com/invitation#\(userId)")

// Generato da Firebase
let dynamicLinksDomain = "https://myqix3appdev.page.link"

// Inizializzo la creazione del DynamicLink
let linkBuilder = DynamicLinkComponents(link: targetLink,
    domainURIPrefix: dynamicLinksDomainURIPrefix)

linkBuilder.iOSParameters = DynamicLinkIOSParameters(bundleID: "...")
linkBuilder.iOSParameters?.appStoreID = "1459917691"
linkBuilder.iOSParameters?.minimumAppVersion = "0.0.2"

linkBuilder.androidParameters = DynamicLinkAndroidParameters()
linkBuilder.androidParameters?.fallbackURL = link

// Crea un versione del link più piccola
linkBuilder.shorten { url, _, error in
    
    guard let url = url, error == nil else {
        Log.error("Error shorting your URL")
        return
    }
    
    // Completo con l'url generato
    completion(url)
}

\end{minted}

\subsection{Ottenere i dati}

Ogni volta che un utente clicca su un dynamicLink iOS controlla
se il link è registrato da qualche app e se questo è il caso
inoltra il link all'app che poi dovrà gestirlo aprendola. 
Il link arriva quindi in una funzione di inizializzazione dell'applicazione all'interno del
file \textbf{AppDelegate}, da cui possiamo analizzare i dati ricevuti e effettuare tutte le operazionni necessarie

\begin{minted}{swift}
// Metodo chiamato allo startup con il dynamic link
func application(_ application: UIApplication,
    continue userActivity: NSUserActivity,
    restorationHandler: @escaping ([UIUserActivityRestoring]?) -> Void) -> Bool {
        
    // Controllo se esiste il link
    if let incomingURL = userActivity.webpageURL {
        // ...
    }
    
    return false
}
\end{minted}