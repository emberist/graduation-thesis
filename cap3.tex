

Avendo definito i principali metodi di navigatione tra ViewController al capitolo~\ref{CH:2} torniamo al problema iniziale:
\textit{Come possiamo rendere dinamica la navigazione?}

A seguito di uno studio approfondito di varie tecniche di navigazione iOS ho scelto di utilizzare il
\textbf{Coordinator Pattern}\cite{coordinatorpattern}.

\section{Il Coordinator Pattern}

Generalmente in iOS l'intera logica di un ViewController viene scritta nel controller stesso, creando spesso
file di grosse dimensioni e disordine generale. Il Coordinator Pattern è nato proprio per rendere 
le applicazioni più scalabili e leggere. 

Ogni ViewController infatti delega tutte le decisioni al suo Coordinator, attraverso il Delegation Pattern (vedere sezione~\ref{delegation}), che in base a determinate logiche deciderà
i passi successivi.

Ogni Coordinator può controllare un ViewController o più Coordinator, questo rende le viste
indipendenti tra di loro e rende ogni ViewControler totalmente invisibile agli altri.\\

\begin{minipage}{\linewidth}
    \centering
    \includegraphics[width=10cm]{coordinator}
    \captionof{figure}{
        Il Coordinator Pattern
    }
    \label{fig:4}
\end{minipage}\\ \\

La resposibilità dei coordinator è infatti la navigazione, come un navigation controller gestisce i sui View Controller, un coordinator gestisce
i suoi figli e questo rende ogni vista o flow di navigazione totalmente indipendente dal resto dell'applicazione.

Per navigare tra i view controller vengono generalmente usate le tipologie di navigazione
descritte nella sezione~\ref{sec:navigation}, tranne le segue, che essendo definete da vista grafica renderebbero
la navigazine statica e fissata su determinati ViewController. \\

Di seguito in figura~\ref{fig:5} presento uno schema dell'utilizzo di due coordinator
per la gestione di una lista di prodotti e il carrello. \\

\begin{minipage}{\linewidth}
    \centering
    \includegraphics[width=10cm]{coordinator-example}
    \captionof{figure}{
       Esempio di coordinator pattern
    }
    \label{fig:5}
\end{minipage}\\ \\

Come si evince dall'immagine è presente in entrambi i coordinator è presente un oggetto
\textbf{navigator} che sarà in gestore di un UINavigationController

\section{Implementazione}

Ogni coordinator ha degli elementi fissi che sono:

\begin{itemize}
    \item Una funzione di partenza denominata \textbf{start};
    \item Un array di coordinator figli;
\end{itemize}

Per questo ho iniziato implementando dei protocolli:
\subsection{Protocolli base}

Il primo protollo implememtato è quello che defisce un qualunque coordinator
e ne tiene in memoria i figli in modo che non vengano
dealloccati automaticamente dal sistema operativo.

\begin{minted}{swift}
protocol Coordinator: class {

    var childCoordinators: [Coordinator] { get set }
    
    /** Utilizzato task di inizializzazione */
    func start()
    
}

extension Coordinator {
    
    // Aggiunge un figlio al coordinator padre
    func add(childCoordinator: Coordinator) {
        childCoordinators.append(childCoordinator)
    }
    
    // Rimuove un Coordinator figlio dal parent
    func remove(childCoordinator: Coordinator) {
        childCoordinators = childCoordinators.filter {
            $0 !== childCoordinator 
        }
    }
}
\end{minted}

Successivamente è stato implementato un BaseCoordinatorPresentable, che estende Coordinator
e aggiunge delle funzionalità di presentazione modale.

\begin{minted}{swift}
protocol BaseCoordinatorPresentable: Coordinator {

    // Il view controller principale del coordinator
    var _rootViewController: UIViewController { get }
}

// MARK: - Presentation Methods

extension BaseCoordinatorPresentable {
    
    /**
     Inizia un coordinator figlio e presenta
     il suo rootViewController modalmente

     - Parametri:
        - childCoordinator: Il coordinator da presentare
        - animated: Specifica se con o senza animazione modale
     */
    
    func presentCoordinator(_ childCoordinator: BaseCoordinatorPresentable,
        animated: Bool) {

        add(childCoordinator: childCoordinator)
        childCoordinator.start()
        _rootViewController.present(childCoordinator._rootViewController, animated: animated)
    }
    
    /**
     Inizia un viewController senza coordinator e
     lo presenta modalmente
     
     - Parametri:
        - controller: Il controller da presentare
        - animated: Specifica se con o senza animazione modale
     */
    
    func present(_ controller: UIViewController, animated: Bool) {
        _rootViewController.present(controller, animated: animated)
    }
    
    /**
     Interrompe la presentazione di un child Coordinator
     eliminandolo anche dall'array in memoria

     - Parameters:
        - childCoordinator: Il coordinator da chiudere e rilasciare
        - animated: Specifica se con o senza animazione modale
        - completion: closure che viene eseguita alla vine del dismiss
     */
    
    func dismissCoordinator(_ childCoordinator: BaseCoordinatorPresentable,
        animated: Bool, completion: (() -> Void)? = nil) {

        childCoordinator._rootViewController.dismiss(animated: animated,
            completion: completion)
        self.remove(childCoordinator: childCoordinator)
    }

}
\end{minted}

\subsection{Coordinator modale}

Il BaseCoordinatorPresentable è stato definito per evitare errori di \textbf{associetedtype},
infatti questo protocollo non deve mai essere implemetato direttamente, ma soltanto con il protocollo sucessivo

\begin{minted}{swift}
protocol CoordinatorPresentable: BaseCoordinatorPresentable {

    /**
        Questo protocolo utilizza la tipizzazione per
        permettere di ottenere Coordinator con view
        controller tipizzati.
     */
    associatedtype ViewController: UIViewController

    // Il rootViewController del coordinator
    var rootViewController: ViewController { get }

}

extension CoordinatorPresentable {

    // Ritorna rootViewController
    var _rootViewController: UIViewController { return rootViewController }
}
\end{minted}

\subsection{Esempio di utilizzo}

\section{Aggiunta del Navigator}