
Il mio compito nello sviluppo dell'applicazione è stato quello 
di creare un \textbf{prototipo} iniziale avendo a disposizione un mock up creato con
proto.io\cite{protoio} e una serie di requisiti essenziali.


La base di partenza di QIX sono state delle funzionalità essenziali e 
sostanzialmente molto difficili da inserire in una versione dell'app già avanzata.
È stato quindi deciso di creare un prototipo di partenza avente i seguenti requisiti:

\begin{itemize}
    \item {
        \textbf{Navigazione dinamica}: L'applicazione deve gestire dei cambiamenti di contesto
        dinamici: dev'essere possibile mostrare all'utente contenuti dinamici indipendentemente
        dal contesto in cui si trova. 
    }
    \item {
        \textbf{QIX Shake}: L'utente deve poter agitare lo smartphone in qualsiasi
        sezione dell'applicazione e il risultato deve essere basato sul contesto attuale o su delle direttive dettate
        da delle Rest API.
    } 
    \item {
        \textbf{Animazioni interattive}: L'intera applicazione dev'essere progettata in modo tale da presentare all'utente
        delle \textbf{animazioni interattive} in stile CardView\cite{cardview} disponibili in 
        qualunque sezione o vista in cui si trovi l'utente e definite dal contesto attuale.

        Le animazioni in questione devono essere progettate in pagine, in cui ogni pagina può contenere 
        più CardView. L'utente vedrà in un determinato momento una e soltanto una pagina.

        Ogni CardView deve essere trascinabile dall'utente e deve interagire con le altre CardView della pagine. 
        Quando l'utente usa una forza di trascinamento superiore a un valore di soglia tutte le viste devono
        cadere per gravità.
        
        Tale gravità finirà con la fine dell'animazione o l'apparizione di una nuova pagina se presente.
    }
    \item {
        \textbf{Autenticazione}: L'applicazione deve supportare tre diversi stati o modalità di autenticazione:
        \begin{enumerate}
            \item\textbf{Trial Mode}: l'utente è anonimo, esiste solo un id per tenere traccia dei suoi QIX coins.
            \item\textbf{Signed Mode}: l'utente ha inserito il numero di telefono e il suo genere.
            \item \textbf{Pro Mode}: l'utente aggiunge dei dati su se stesso o collega il suo account a dei social media come Facebook, Google o Instagram.
        \end{enumerate}
        Si nota facilmente che non esiste una stato in cui l'utente non è registrato: questo perchè
        per tenere traccia dei suoi QIX coins e di altri dati utili è necessario avere una riferimento all'utente;
    }
    \item {
        \textbf{DeepLinks}: L'applicazione deve poter essere avviata dinamicamente
        attraverso dei \textbf{Deep Links}\cite{deeplinks}.
        E deve essere in grado di gestirli in base al contesto dell'utente.
    }
\end{itemize}