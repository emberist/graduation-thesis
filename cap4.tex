
\section{Cos'è il QIX Shake?}

In iOS ogni UIViewController risponde a degli eventi. L'evento designato per lo shake è
\begin{minted}{swift}
    func motionEnded(_ motion: UIEvent.EventSubtype,
        with event: UIEvent?)
\end{minted}

In caso di shake infatti motion sarà uguale a .motionShake. \\

Inizialmente per rendere disponibile l'evento "shake" in un qualunque ViewController la soluzione è stata abbastanza semplice:
è bastato l'utilizzo di un ViewController genitore e attraverso l'ereditarietà ogni view controller è stato in grado
di eseguire la stessa funzionalità.

Successivimente sono stati riscontrati dei problemi, dato che con il sempre più altro grado di complessità
ogni vista avrebbe dovuto ereditare una classe base. In alcuni casi questo non è stato possibile. Ho quindi optato 
per un'\textbf{estensione} della classe UIViewController e un semplice protocollo che aggiunge agli elementi che lo ereditono
una closure opzionale.

Per notificare la motion del QIX Shake ho utilizzato delle notifiche locali: quando avviene uno shake il ViewController interessato invia 
una notifica globale e, nel mio caso, solamente l'AppCoordinator riceverà tale notifica.
Si nota anche che questa notifica contiene un campo \textbf{sender} che indentifica il ViewController da cui viene lanciata e quindi
il \textbf{contesto attuale} dell'utente.

\section{Implementazione sfruttando ereditarietà}

\begin{minted}{swift}
class BaseViewController: UIViewController {

    override open func motionBegan(_ motion:
        UIEvent.EventSubtype, with event: UIEvent?) {
            
        if motion == .motionShake {
            let notification = Notification(
                name: Notification.Name.UserDidShake,
                userInfo: ["sender": self]
            )
            NotificationQueue.default.enqueue(notification,
                postingStyle: .asap)
        }
    }
}
\end{minted}

Sebbene questo metodo sia semplice e funzioni, quello mostrato di seguita
risulta migliore per gli scopi voluti.

% \newpage
\section{Implementazione con estensione}

\begin{minted}{swift}
protocol Shakerable {
    var shakeAction: ShakeClousure? { get }
}

extension UIViewController {
   
    override open func motionBegan(_ motion:
        UIEvent.EventSubtype, with event: UIEvent?) {

        if motion == .motionShake {
            let notification = Notification(
                name: Notification.Name.UserDidShake,
                userInfo: ["sender": self]
            )
            
            NotificationQueue.default.enqueue(notification,
                postingStyle: .asap)
            
            if let shakerable = self as? Shakerable {
                shakerable.shakeAction?(self)
            }
        }
    }
}
\end{minted}

L'esempio riportato qui sopra è molto simile a quello precedente, ma utilizzando le estensioni
ogni UIViewController erediterà questo metodo sovrascritto senza dover far ereditare a tutti i ViewController
una classe aggiuntiva, anche perchè, come abbiamo definito in precedenza, il QIX Shake deve essere disponibile
ovunque. 

Un altro fattore interessante è il protocollo \textbf{Shakerable} dato che tutti i ViewController che lo implementano
possono, attraverso la shakeAction closure, eseguire ciò che devono a seguito di uno shake.